\begin{abstract}
Huge volumes of data are generated every day at high speeds.
In addition, these data tend to come from very varied sources
and therefore have heterogeneous structures. RDF was proposed as a
graph-based data model. It relies on the global nature
of URIs to solve the integration of produced data
by different sources. However RDF by itself does not allow to validate
that a graph complies with a contract or structure. That is why the
technology called Shape Expressions was born. Through Shape Expressions
an RDF graph can be validated against a certain schema.
So with a set of Shape Expressions you can
describe a domain model of a RDF data set. This technology is supported by an active community that has developed
tools such as development environments, RDF validators, inference systems, etc ...

In this work, on one hand, a static analysis system is proposed. From which
up to 90\% of current Shape Expressions users can benefit. It is based on
lexical, syntactic and semantic analysis that improves the detection and reporting system for
errors and warnings. On the other hand, as Shape Expressions has as
purpose to model a domain and this is the same purpose as what the
object-oriented programming languages have. We provide a System to automatically
translate shape expressions into object models of different object oriented languages.
With that system, we have been able to transform 50\% of all existing shape expressions
in GitHub to Java classes. The translated shapes are those ones that are syntactically valid
and stick to some language restrictions.

\bigskip
\textbf{Keywords ---} \textit{RDF, Linked Data, RDF Validation, Shape Expressions, Lexical-Syntactic and
Semantic Validator, Object Oriented Programming Languages, Compiler, Translator.}
\end{abstract}