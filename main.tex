%%%%%%%%%%%%%%%%%%%%%%%%%%%%%%%%%%%%%%%%%
% kaobook
% LaTeX Template
% Version 1.2 (4/1/2020)
%
% This template originates from:
% https://www.LaTeXTemplates.com
%
% For the latest template development version and to make contributions:
% https://github.com/fmarotta/kaobook
%
% Authors:
% Federico Marotta (federicomarotta@mail.com)
% Based on the doctoral thesis of Ken Arroyo Ohori (https://3d.bk.tudelft.nl/ken/en)
% and on the Tufte-LaTeX class.
% Modified for LaTeX Templates by Vel (vel@latextemplates.com)
%
% License:
% CC0 1.0 Universal (see included MANIFEST.md file)
%
%%%%%%%%%%%%%%%%%%%%%%%%%%%%%%%%%%%%%%%%%

%----------------------------------------------------------------------------------------
%	PACKAGES AND OTHER DOCUMENT CONFIGURATIONS
%----------------------------------------------------------------------------------------

\documentclass[
	fontsize=11pt, % Base font size
	twoside=true, % Use different layouts for even and odd pages (in particular, if twoside=true, the margin column will be always on the outside)
	%open=any, % If twoside=true, uncomment this to force new chapters to start on any page, not only on right (odd) pages
	%chapterprefix=true, % Uncomment to use the word "Chapter" before chapter numbers everywhere they appear
	%chapterentrydots=true, % Uncomment to output dots from the chapter name to the page number in the table of contents
	numbers=noenddot, % Comment to output dots after chapter numbers; the most common values for this option are: enddot, noenddot and auto (see the KOMAScript documentation for an in-depth explanation)
	%draft=true, % If uncommented, rulers will be added in the header and footer
	%overfullrule=true, % If uncommented, overly long lines will be marked by a black box; useful for correcting spacing problems
]{kaobook}

% Set the language
\usepackage[english]{babel} % Load characters and hyphenation
\usepackage[english=british]{csquotes} % English quotes

% Load packages for testing
\usepackage{blindtext}
%\usepackage{showframe} % Uncomment to show boxes around the text area, margin, header and footer
%\usepackage{showlabels} % Uncomment to output the content of \label commands to the document where they are used

% Load the bibliography package
\usepackage{styles/kaobiblio}
\addbibresource{main.bib} % Bibliography file

% Load mathematical packages for theorems and related environments. NOTE: choose only one between 'mdftheorems' and 'plaintheorems'.
\usepackage{styles/mdftheorems}
%\usepackage{styles/plaintheorems}

\graphicspath{{examples/documentation/images/}{images/}} % Paths in which to look for images

\makeindex[columns=2, title=Alphabetical Index, intoc] % Make LaTeX produce the files required to compile the index

\makeglossaries % Make LaTeX produce the files required to compile the glossary

\makenomenclature % Make LaTeX produce the files required to compile the nomenclature

% Reset sidenote counter at chapters
%\counterwithin*{sidenote}{chapter}

%----------------------------------------------------------------------------------------

\begin{document}

%----------------------------------------------------------------------------------------
%	BOOK INFORMATION
%----------------------------------------------------------------------------------------

\titlehead{ShEx-Lite}
\subject{Final Degree Project}

\title[ShEx-Lite]{ShEx-Lite}
\subtitle{Automatic generation of domain object models through a subset of a Shape Expressions Compact Syntax.}

\author[Guillermo Facundo Colunga]{Guillermo Facundo Colunga}

\date{\today}

\publishers{School of Computer Science\\University of Oviedo}

%----------------------------------------------------------------------------------------

\frontmatter % Denotes the start of the pre-document content, uses roman numerals

%----------------------------------------------------------------------------------------
%	OPENING PAGE
%----------------------------------------------------------------------------------------

%\makeatletter
\includepdf{images/disertation-official-page.pdf}
%\makeatother

%----------------------------------------------------------------------------------------
%	COPYRIGHT PAGE
%----------------------------------------------------------------------------------------

\makeatletter
\uppertitleback{\@titlehead} % Header

\lowertitleback{
	Final Degree Project presented on July 2020 at the School of Software Engineering, Oviedo University. All the source code related to the implementation explored during this book is available at \url{github.com/weso/shex-lite}.

	\medskip

	\textbf{Copyright}\\
	All rights reserved. This project or any portion may not be reproduced or used in any manner without the express quotation to the original author.

	\medskip

	\textbf{Directors} \\
	Dr. Jose Emilio Labra Gayo\\Daniel Fernández Álvarez
}
\makeatother

%----------------------------------------------------------------------------------------
%	DEDICATION
%----------------------------------------------------------------------------------------

\dedication{
	The harmony of the world is made manifest in Form and Number, and the heart and soul and all the poetry of Natural Philosophy are embodied in the concept of mathematical beauty.\\
	\flushright -- D'Arcy Wentworth Thompson
}

%----------------------------------------------------------------------------------------
%	OUTPUT TITLE PAGE AND PREVIOUS
%----------------------------------------------------------------------------------------

% Note that \maketitle outputs the pages before here

% If twoside=false, \uppertitleback and \lowertitleback are not printed
% To overcome this issue, we set twoside=semi just before printing the title pages, and set it back to false just after the title pages
\KOMAoptions{twoside=false} % default as 'semi' but turned to false so the first page is printed right after the cover, the cover is intended to be prented separetly.
\maketitle
\KOMAoptions{twoside=false}

%----------------------------------------------------------------------------------------
%	AKNOWLEDGEMENTS
%----------------------------------------------------------------------------------------

\chapter*{Aknowledgments}
\addcontentsline{toc}{chapter}{Aknowledgments} % Add this section to the table of contents as a chapter
// TODO

\begin{kaobox}[frametitle=Spanish]
	// TODO
\end{kaobox}

%----------------------------------------------------------------------------------------
%	ABSTRACT
%----------------------------------------------------------------------------------------

\chapter*{Abstract}
\addcontentsline{toc}{chapter}{Abstract} % Add the preface to the table of contents as a chapter

This end of degree project is about creating a compiler for a subset of the Shape Expressions Compact Syntax, focused on syntactic and semantic validation and the generation of domain models in object oriented languages. \todo{Completar...}


%----------------------------------------------------------------------------------------
%	TABLE OF CONTENTS & LIST OF FIGURES/TABLES
%----------------------------------------------------------------------------------------

\begingroup % Local scope for the following commands

% Define the style for the TOC, LOF, and LOT
%\setstretch{1} % Uncomment to modify line spacing in the ToC
%\hypersetup{linkcolor=blue} % Uncomment to set the colour of links in the ToC
\setlength{\textheight}{23cm} % Manually adjust the height of the ToC pages

% Turn on compatibility mode for the etoc package
\etocstandarddisplaystyle % "toc display" as if etoc was not loaded
\etocstandardlines % toc lines as if etoc was not loaded

\tableofcontents % Output the table of contents

\listoffigures % Output the list of figures

% Comment both of the following lines to have the LOF and the LOT on different pages
\let\cleardoublepage\bigskip
\let\clearpage\bigskip

\listoftables % Output the list of tables

\endgroup

%----------------------------------------------------------------------------------------
%	MAIN BODY
%----------------------------------------------------------------------------------------

\mainmatter % Denotes the start of the main document content, resets page numbering and uses arabic numbers
\setchapterstyle{kao} % Choose the default chapter heading style

\setchapterpreamble[u]{\margintoc}
\chapter{Introduction}
\labch{intro}

\textit{This project was born in the bar of a pub where the parents of RDF graph validation were talking about creating a tool that would allow people who were not computer-scientist to get started and later work with RDF graph validation schemes.}

% The motivation section.
\section{Motivation}\labsec{ch01-motivation}

Each day more and more devices generate data both automatically and manually, and also each day the development of application in different domains that are backed by databases and expose these data to the web becomes easier. The amount and diversity of data produced clearly exceeds our capacity to consume it.

To describe the data that is so large and complex that traditional data processing applications can’t handle the term big data \sidecite[-100pt]{big-data} has emerged. Big data has been described by at least three words starting by V: volume, velocity, variety. Although volume and velocity are the most visible features, variety is a key concept which prevents data integration and generates lots of interoperability problems.

In order to solve this key concept RDF (Resource Description Framework) was proposed as a graph-based data model \sidecite[-150pt]{graph-data-model} which became part of the Semantic Web \sidecite[-125pt]{semantic-web} vision. Its reliance on the global nature of URIs\sidenote[][-100pt]{A Uniform Resource Identifier (URI) is a string of characters that unambiguously identifies a particular resource. To guarantee uniformity, all URIs follow a predefined set of syntax rules, but also maintain extensibility through a separately defined hierarchical naming scheme.\\Ref.\url{https://en.wikipedia.org/wiki/Uniform_Resource_Identifier}} offered a solution to the data integration problem as RDF datasets produced by different means can seamlessly be integrated with other data.

Also, and related to his is the concept of Linked Data \sidecite{linked-data} that was proposed as a set of best practices to publish data on the Web. It was introduced by Tim Berners-Lee and was based on four main principles:

\begin{itemize}
  \item Use URIs as names for things.
  \item Use HTTP URIs so that people can look up those names.
  \item When someone looks up a URI, provide useful information, using the standards (RDF, SPARQL).
  \item Include links to other URIs. so that they can discover more things.
\end{itemize}

\begin{marginfigure}
	\includegraphics{5-star-steps}
	\caption[The 5 star steps of Linked Data]{The 5 star steps of Linked Data.}
	\labfig{margin-5-star-steps}
\end{marginfigure}

This four principles are called the 5 stars Linked Open Data Model, illustrated in \reffig{margin-5-star-steps}.
RDF is mentioned in the third principle as one of the standards that provides useful information. The goal of this principles is that data is not only ready for humans to navigate through but also for other agents, like computers, that may automatically process that data.

All the above motivations helped to make RDF the language for the Web of Data, as described in \sidecite{labra-validating-rdf}. And the main features that it presents are: Disambiguation, Integration, Extensibility, Flexibility and Open by Default. All this concepts will be deeply explored in \refsec{ch02-rdf}, but with the features also some drawbacks are associated, the most important one and the one we will focus is the RDF production/consumption dilema.

RDF production/consumption dilema states that it is necessary to find ways that data producers can generate their data so it can be handled by potential consumers. For example, they may want to declare that some nodes have some properties with some specific values. Data consumers need to know that structure to develop applications to consume the data.

Although RDF is a very flexible schema-less language, enterprise and industrial applications may require an extra level of validation before processing for several reasons like security, performance, etc.

To solve that dilema and as an alternative to expecting the data to have some structure without validation, Shape Expressions (ShEx) where proposed as a human-readable and high-level open source language for RDF validation. Initially ShEx was proposed as a human-readable syntax for OSLC Resource Shapes \sidecite{oslc-resource-shape} but ShEx grew very fast to embrace more complex user requirements coming from clinical and library use cases.

Another technology, SPIN, was used for RDF validation, principally in TopQuadrant’s TopBraid Composer. This technology, influenced from OSLC Resource Shapes as well, evolved into both a private implementation and open source definition of the Shapes Constraint Language (SHACL), which was adopted by the W3C Data Shapes Working Group.

From a user point of view the possibilities of ShEx are very large, from the smallest case to just validate a node with one property to a scientific domain case where we need to validate the human genome (a real use case of ShEx). As seen, ShEx is a new powerful language, but it can became complicated on the corner cases, but most of day-to-day uses can be solved with a subset of the language. This is the point where this project borns. We will call this subset ShEx-Lite. The simplicity of ShEx-Lite is not only focus on computer scientists who have experience the pain of new languages but also for other non-technical profiles that need to validate RDF data.

Besides to this, a common problem is that some companies use ShEx to define the constraints of the RDF data that they own. But then, when developing applications with object oriented languages they need to translate those schemas in to a domain model to support their data. Furthermore if the Shape Expressions used to validate their data changes for some reason they need to rewrite that domain model in the OOL again.

Finally, from a ShEx developer point of view sometimes appears the need to try new features in a small playground that allow easy an fast testing.


% The purpose section.
\section{Main usage scenarios}\labsec{ch01-main-usage-scenarios}

\refsec{ch01-motivation} introduced some profiles that might benefit from using ShEx-Lite. We can find an example of this profiles in the Wikidata Community. Wikidata is formed by a multidisciplinar community whose aim is to introduce RDF data in to an open knowledge base used by other companies like Google Search. The only problem is that the introduced RDF data needs validation to ensure a minimum data quality, but the profiles that introduce the data, usually, are domain experts whose knowledge about computer science is limited. \todo{Extender ejemplo wikidata.}

Besides to this, a common problem is that some companies like Wikidata or even Universities use ShEx to define the constraints of the RDF data that they own. But then, when developing applications with object oriented languages they need to translate those schemas in to a domain model to support their data. Furthermore if the Shape Expressions used to validate their data changes for some reason they need to rewrite that domain model in the OOL again. \todo{Adornar un poco.}

Finally, from a ShEx developer point of view sometimes appears the need to try new features in a small playground that allow easy an fast testing, for example a feature that appeared after this project was implemented is to automatically generate documentation webpages for the schemas defined in ShEx, but the first target of this feature won’t be ShEx, will be ShEx-Lite as it is perfect for he proof of concept.

\todo{Enumerar tipos de erramientas que se beneficiarían.}


% The contents of the proposal section.
\section{Content of the proposal}\labsec{ch01-content-of-the-proposal}

After \refsec{ch01-motivation} and \refsec{ch01-main-usage-scenarios} this section descibes the developed system to solve the deficiencies and different profile-users requests.

\begin{description}
  \item[First] A compiler for a language defined as a subset of the shape expressions language focused on helping the non-expert user on solving problems with their schemas.
  \item[Secondly] A functionality in this compiler, that allows to automatically create domain object models in object-oriented programming languages, from the defined schemas.
\end{description}

\subsection{ShEx-Lite Compiler}
This compiler works over a defined subset of the Shape Expressions Compact Syntax, defined at \sidecite{shexc} that allows expressing basic constraints. It is implemented with the paradigm "compiler as a library" and it is able to parse a schema, analyze it and generate the syntactic and semantic errors that the schema contains.

The ShEx-Lite Compiler is composed of the following components:

\subsubsection{Syntax analysis}
The syntax analysis phase covers the transformation of the input in to an Abstract Syntax Tree. That is, lex and parse the file, generate the parse tree, raise any errors or warnings and finally build the AST.

\subsubsection{Semantic analysis}
The semantic analysis covers the validation and transformation of the AST in to the SIL (ShEx-Lite Intermediate Language). Is during this stage where the AST gets validated, type-checked and transformed from a tree to a graph, is this graph the one that gets the name of SIL.




\subsection{Automatic generation of domain object models}
But by far, the biggest difference with existing tools, is the automatic generation of domain object models from the schemas defined.

The idea behind this is to enhance interoperability between object oriented languages \sidecite[-40pt]{oopl} and RDF systems. An example of this is the European Project ASIO Hércules \sidecite[-20pt]{hercules-um}, where the automatic transformation of schemas in to POJOs \sidecite{pojo} is the tool that joins the Semantic Architecture and the Ontology Infrastructure.

Also it is important to remark here that we are perfectly conscious about the fact that not every object oriented language allows to model exactly the same restrictions as types differ, therefore each OOL needs to validate or map the schema to a representation on the language whose meaning is the same, that is create the image of the schema in the corresponding language.


% The contents section.
\section{Structure}\labsec{ch01-structure}
The dissertation layout is as follows:\\

\begin{description}
	\item[Chapter 2] Indicates the state of the art of the existing RDF validation technologies, tools for processing Shape Expressions and other related projects.
	\item[Chapter 3] Describes the goals that the project aim to achieve after its execution and possible real-world applications.
	\item[Chapter 4] Contains a detailed initial planning and budget for the project, this is the designed planning followed during the execution of the project and the initial estimated budget.
	\item[Chapter 5] Gives a basic theoretical background that it is needed to fully understand the concepts explained in the following chapters.
	\item[Chapter 6] Provides a technical description of the design and implementation of the compiler itself. This includes, analysis, design, the technological stack choices, diagrams, implementation decisions and tests.
	\item[Chapter 7] Compares the initial planning developed in chapter 4 with the final one. This includes the genuine execution planning of the project and the reasons and events that modified the one from chapter 4.
	\item[Chapter 8] Summarizes the analysis and results given over the project, gives an outlook for future work continuing the development of the implemented solution. And includes the diffusion of results done during the project.
	\item[Chapter 9] Includes all the set of references used during this document. It is fully recommended to read them carefully and use them as source of truth for any doubt.
	\item[Chapter 10] Attaches every document related to the project and referenced from other chapters that has been developed during the project. Here we include detailed budget, system manuals, and other documents.
\end{description}


\setchapterpreamble[u]{\margintoc}
\chapter{Theoretical Background}
\labch{theory}

For a proper understanding of this documentation and the ideas explained on it it is needed to know some theoretical concepts that are the fundaments of Linked Data, RDF, RDF Validation, programing languages and compilers. This sections is devoted to carefully explain those concepts to the needed deepth to fully understand this dissertation, but for those readers that want a deeper explanation a more detailed view of the concepts presented here is offered in \sidecite{labra-validating-rdf, eric-rdf-validation-lang, programing-language}.

% Section 1, RDF.
\section{RDF}
\labsec{ch02-rdf}
Resource Description Framework (RDF) is a standard model for data interchange on the web, started in 1998 and the first version of the specification was published in 2004 by the W3C according to \sidecite{rdf-primer}. RDF has features that facilitate data merging even if the underlying schemas differ, and it specifically supports the evolution of schemas over time without requiring all the data consumers to be changed. Another important feature is that RDF supports XML, N-Triples and Turtle syntax, the \reffig{rdf-ntriples-ex} shows an example of how a triplet can be written in RDF N-Triples Syntax.

\begin{figure}[hb]
\begin{lstlisting}
<http://example/subject1> <http://example/predicate1> <http://example/object1>
\end{lstlisting}
\caption[RDF N-Triples Example]{RDF N-Triples Example. From this example we can see that each triplet is composed of three elements, the subject the predicate and the object.}
\labfig{rdf-ntriples-ex}
\end{figure}

RDF extends the linking structure of the Web to use URIs to name the relationship between things as well as the two ends of the link (this is usually referred to as a “triple” or "triplet"). Using this simple model, it allows structured and semi-structured data to be mixed, exposed, and shared across different applications. \reffig{rdf-graph} shows an example of how different triples can be use to compose a graph, this graph represents the same as the \reffig{rdf-ntriples-graph}

\begin{figure}[hb]
\begin{lstlisting}
<http://example/bob> <http://example/knows> <http://example/alice> .
<http://example/alice> <http://example/knows> <http://example/peter> .
\end{lstlisting}
\caption[RDF N-Triples Graph Example]{RDF N-Triples Graph Example. This exmaple shows the n-triples that generate the graph from \reffig{rdf-graph}.}
\labfig{rdf-ntriples-graph}
\end{figure}

This linking structure forms a directed, labeled graph, where the edges represent the named link between two resources, represented by the graph nodes. This graph view is the easiest possible mental model for RDF and is often used in easy-to-understand visual explanations.

Also, related to this we strongly recommend the Tim Berners-Lee’s writings on Web Design Issues \sidecite{semantic-roadmap} where he explain the issues of the liked data and why is RDF so important.

\begin{marginfigure}
\includegraphics{rdf-graph}
\caption[RDF Example graph]{RDF Example graph.}
\labfig{rdf-graph}
\end{marginfigure}


% Section 2, Validating RDF.
\section{Validating RDF}
In the previous point we just see that easiest possible mental model for RDF is a graph, and that’s correct. At the end RDF represents a graph. And with the ability of representing and storing data emerges the need to validate that the schema of the graph is correct.
At the time RDF was introduced, as it was based on XML the usage of XML-Schema. But this was a very esoteric way of doing it. In order to solve this problem in 2017 the book Validating RDF Data \sidecite{gayo2017validating} was published.
In the book different alternatives for validating RDF are explained under Chapters 3, 4 and 5. But also describes the possible applications of RDF validation in Chapter 6 and finally in Chapter 7 they make an small comparison about RDF validation technologies.
The most wide used validation technology is Shape Expressions even though the W3C standard points to SHACL which is was based in the Shape Expressions: An RDF validation and transformation language \sidecite{prud2014shape} paper. That is the main reason why this technology will be the one that we will explore deeply.

\subsection{Shape Expressions} \todo{Link references.}
As defined in [2] Shape Expressions (ShEx) is a schema language for describing RDF graphs structures. ShEx was originally developed in late 2013 to provide a human-readable syntax for OSLC Resource Shapes. It added disjunctions, so it was more expressive than Resource Shapes. Tokens in the language were adopted from Turtle and SPARQL with tokens for grouping, repetition and wildcards from regular expression and RelaxNG Compact Syntax [16]. The language was described in a paper [1] and codified in a June 2014 W3C member submission [17] which included a primer and a semantics specification. This was later deemed “ShEx 1.0”.
The W3C Data Shapes Working group started in September 2014 and quickly coalesced into two groups: the ShEx camp and the SHACL camp. In 2016, the ShEx camp split from the Data Shapes Working Group to form a ShEx Community Group (CG). In April of 2017, the ShEx CG released ShEx 2 with a primer, a semantic specification and a test-suite with implementation reports.
As of publication, the ShEx Community Group was starting work on ShEx 2.1 to add features like value comparison and unique keys. See the ShEx Homepage \url{http://shex.io/} for the state of the art in ShEx. A collection of ShEx schemas has also been started at \url{https://github.com/shexSpec/schemas}.

\begin{figure}[hb]
\begin{lstlisting}
PREFIX :       <http://example.org/>
PREFIX schema: <http://schema.org/>
PREFIX xsd:  <http://www.w3.org/2001/XMLSchema#>

:User {
  schema:name          xsd:string  ;
  schema:birthDate     xsd:date?  ;
  schema:gender        [ schema:Male schema:Female ] OR xsd:string ;
  schema:knows         IRI @:User*
}
\end{lstlisting}
\caption[Shape Expression Example]{Shape Expression Example. This example describes a shape expression that describes a user as a node that has one name of type string, an optional bithd date of type date, one gender of type Male, Female or free string and a set between 0 and infinite of other users represented by the knows property.}
\labfig{shape-expr-ex}
\end{figure}

\subsubsection{Use of ShEx}
Strictly speaking, a ShEx schema defines a set of graphs. This can be used for many purposes, including communicating data structures associated with some process or interface, generating or validating data, or driving user interface generation and navigation. At the core of all of these use cases is the notion of conformance with schema. Even one is using ShEx to create forms, the goal is to accept and present data which is valid with respect to a schema.
ShEx has several serialization formats:

\begin{itemize}
	\item a concise, human-readable compact syntax (ShExC);
	\item a JSON-LD syntax (ShExJ) which serves as an abstract syntax; and
	\item an RDF representation (ShExR) derived from the JSON-LD syntax.
\end{itemize}

These are all isomorphic and most implementations can map from one to another.
Tools that derive schemas by inspection or translate them from other schema languages typically generate ShExJ. Interactions with users, e.g., in specifications are almost always in the compact syntax ShExC. As a practical example, in HL7 FHIR, ShExJ schemas are automatically generated from other formats, and presented to the end user using compact syntax. See Section 6.2.3 for more details.
ShExR allows to use RDF tools to manage schemas, e.g., doing a SPARQL query to find out whether an organization is using dc:creator with a string, a foaf:Person, or even whether an organization is consistent about it.

\subsubsection{ShEx Implementations} \todo{Check links.}
At the time of this writing, we are aware of the following implementations of ShEx.

\begin{itemize}
	\item shex.js for Javascript/N3.js (Eric Prud’hommeaux) \url{https://github.com/shexSpec/shex.js/};
	\item Shaclex for Scala/Jena (Jose Emilio Labra Gayo) \url{https://github.com/labra/shaclex/};
	\item shex.rb for Ruby/RDF.rb (Gregg Kellogg) \url{https://github.com/ruby-rdf/shex};
	\item Java ShEx for Java/Jena (Iovka Boneva/University of Lille) \url{https://gforge.inria.fr/projects/shex-impl/}; and
	\item ShExkell for Haskell (Sergio Iván Franco and Weso Research Group) \url{https://github.com/weso/shexkell}.
\end{itemize}

There are also several online demos and tools that can be used to experiment with ShEx.

\begin{itemize}
	\item shex.js (http://rawgit.com/shexSpec/shex.js/master/doc/shex-simple.html);
	\item Shaclex (http://shaclex.herokuapp.com); and
	\item ShExValidata (for ShEx 1.0) (https://www.w3.org/2015/03/ShExValidata/).
\end{itemize}

\subsection{Other Technologies}
As other validation technologies we will just explore the existence of them as it is very interesting to know how other tools approach the same issue.

\subsubsection{SHACL}
Also in [2], Chapter 5, it is fully explained that Shapes Constraint Language (SHACL) has been developed by the W3C RDF Data Shapes Working Group, which was chartered in 2014 with the goal to “produce a language for defining structural constraints on RDF graphs [18].”
The main difference that made us choose ShEx over SHACL are that ShEx emphasized human readability, with a compact grammar that follows traditional language design principles and a compact syntax evolved from Turtle.

\subsubsection{JSON Schema}
JSON Schema born as a way to validate JSON-LD, and as turtle and RDF can be serialized as JSON-LD it is usual to think that JSON Schema can validate RDF data, but this is not fully correct. And the reason is that the serialization of RDF data in to JSON-LD is not deterministic, that means that a single schema might have multiple serializations, which interferes with the validation as you cannot define a relative schema.


% Section 3, Programming Languages.
\section{Programming Languages}
According to \sidecite{programing-language} “a programming language is a formal language comprising a set of instructions that produce various kinds of output.” When we talk about programming languages we need to know that they are split into two, General Purpose Languages (GPL) and Domain Specific Languages (DSL). The main difference overtime is that, as said in \sidecite{dsl}, a domain-specific language (DSL) is a computer language specialized to a particular application domain in contrast to a general-purpose language (GPL), which is broadly applicable across domains.
In the specific case of ShEx-Lite we will be talking about a Domain Specific Language, and more deep we would classified it as a Declarative one, that means that it is not Touring Complete \sidecite{touring-complete}.


% Section 4, Compilers.
\section{Compilers}
A compiler is a computer program that translates computer code written in one programming language (the source language) into another language (the target language). Is during this translation process where the compiler validates the syntax and the semantics of the program, if any error is detected in the process the compiler raises an exception (understand as a compiler event that avoids the compiler to continue its execution).

\subsection{Internal Structure}
In order to decompose the internal structure of a compiler they have been split in to the most common task they do \reffig{compiler-stages}, of course this doesn’t mean that there are compilers with more or less stages, but at the end everything can be group into any of the groups that we will explain:

\begin{figure}[hb]
  \includegraphics[width=50mm,scale=0.5]{compiler-stages}
  \caption[Compiler stages]{Compiler stages.}
  \labfig{compiler-stages}
\end{figure}

\subsubsection{Lexycal Analyzer}
The lexical analyzer task is to get the input and split it in to tokens \sidecite{lexical-analysis}, which are build from lexemes. If the compiler cannot find a valid token for some lexemes in the source code will generate an error, as the input cannot be recognized.

\subsubsection{Syntactic Analyzer}
The syntactic analyzer takes the tokens generated during the lexical analysis and parses them in such a way that try’s to group tokens so the conform to the language grammar rules. During this stage if there is any error while trying to group the tokens then the compiler will rise an error as the input cannot be parsed.

\subsubsection{Semantic Analyzer}
The semantic analyzer has two main tasks, usually. First it validates that the source code semantics are correct, for example 4 + “aaa” would not make sense. And the second task is to transform the Abstract Syntax Tree in to a type-checked and annotated AST. Usually that means relate the invocations and variables to its definition, very useful for type-checking.

\subsubsection{Code Generator}
The task of the code generator as its name indicates is to generate the target code, it can be byte code, machine code or even another high-language code.

\subsubsection{Code Optimizer}
The code optimizer is the last step before the final target code is generated, it rewrites the code that the code generator produced without changing the semantics of the program, its aim is just to make code faster. At \sidecite{compiler-optimizations} you can see an example of some optimizations that can be done at compile time to make your code faster.

\subsection{Conventional Compilers}
Conventional compiler are a big monolith where each stage \reffig{compiler-stages} is executed automatically after the previous stage, if the compiler has eight steps you need to execute them all at once. This approach have been the “old-fashion” but it presents some drawbacks:
\begin{itemize}
	\item A poor IDE \sidecite{ide} integration. IDE’s need to perform incremental compilations in matter of nanoseconds so the user doesn’t feel lag when typing the program. With conventional compilers as you need to go through all the compilation process at once they where very slow and companies like Microsoft need to develop different compilers, one for the IDE and another for the final compilation of the program itself. This lead to several problems like that if a feature gets implemented in the final compilation compiler but not in the IDE one the IDE would not support the feature meanwhile the language would.
	\item Difficult to debug. As the conventional compilers where a blackbox the only way to test intermediate stages was by throwing an input and waiting the the feature you wanted to test was thrown for that input.
\end{itemize}

\subsection{Modern Compilers}
After the problems Microsoft had with the C\# compiler they decide to rewrite the whole compiler and introduce a concept called “compiler as an API” with Roslyn \sidecite{dotNet}. This concept has been perfectly accepted and solved many problems. In this concept each stage has an input and an output that can be accessed from outside the compiler and stages can be executed independently on demand. This means that for example if an IDE just want to execute the Lexer the Parser and the Semantic analysis it can. That translates in to speed for the user.

Also the second problem is solved as testing individual parts of the compiler is much more easy than the hole compiler at once.



\appendix % From here onwards, chapters are numbered with letters, as is the appendix convention

\pagelayout{wide} % No margins
\addpart{Annexes}
\pagelayout{margin} % Restore margins

\input{chapters/appendix.tex}

%----------------------------------------------------------------------------------------

\backmatter % Denotes the end of the main document content
\setchapterstyle{plain} % Output plain chapters from this point onwards

%----------------------------------------------------------------------------------------
%	BIBLIOGRAPHY
%----------------------------------------------------------------------------------------

% The bibliography needs to be compiled with biber using your LaTeX editor, or on the command line with 'biber main' from the template directory

\defbibnote{bibnote}{Here are the references in citation order.\par\bigskip} % Prepend this text to the bibliography
\printbibliography[heading=bibintoc, title=Bibliography, prenote=bibnote] % Add the bibliography heading to the ToC, set the title of the bibliography and output the bibliography note

%----------------------------------------------------------------------------------------
%	NOMENCLATURE
%----------------------------------------------------------------------------------------

% The nomenclature needs to be compiled on the command line with 'makeindex main.nlo -s nomencl.ist -o main.nls' from the template directory

\nomenclature{$c$}{Speed of light in a vacuum inertial frame}
\nomenclature{$h$}{Planck constant}

\renewcommand{\nomname}{Notation} % Rename the default 'Nomenclature'
\renewcommand{\nompreamble}{The next list describes several symbols that will be later used within the body of the document.} % Prepend this text to the nomenclature

\printnomenclature % Output the nomenclature

%----------------------------------------------------------------------------------------
%	GREEK ALPHABET
% 	Originally from https://gitlab.com/jim.hefferon/linear-algebra
%----------------------------------------------------------------------------------------

\vspace{1cm}

{\usekomafont{chapter}Greek Letters with Pronounciation} \\[2ex]
\begin{center}
	\newcommand{\pronounced}[1]{\hspace*{.2em}\small\textit{#1}}
	\begin{tabular}{l l @{\hspace*{3em}} l l}
		\toprule
		Character & Name & Character & Name \\
		\midrule
		$\alpha$ & alpha \pronounced{AL-fuh} & $\nu$ & nu \pronounced{NEW} \\
		$\beta$ & beta \pronounced{BAY-tuh} & $\xi$, $\Xi$ & xi \pronounced{KSIGH} \\
		$\gamma$, $\Gamma$ & gamma \pronounced{GAM-muh} & o & omicron \pronounced{OM-uh-CRON} \\
		$\delta$, $\Delta$ & delta \pronounced{DEL-tuh} & $\pi$, $\Pi$ & pi \pronounced{PIE} \\
		$\epsilon$ & epsilon \pronounced{EP-suh-lon} & $\rho$ & rho \pronounced{ROW} \\
		$\zeta$ & zeta \pronounced{ZAY-tuh} & $\sigma$, $\Sigma$ & sigma \pronounced{SIG-muh} \\
		$\eta$ & eta \pronounced{AY-tuh} & $\tau$ & tau \pronounced{TOW (as in cow)} \\
		$\theta$, $\Theta$ & theta \pronounced{THAY-tuh} & $\upsilon$, $\Upsilon$ & upsilon \pronounced{OOP-suh-LON} \\
		$\iota$ & iota \pronounced{eye-OH-tuh} & $\phi$, $\Phi$ & phi \pronounced{FEE, or FI (as in hi)} \\
		$\kappa$ & kappa \pronounced{KAP-uh} & $\chi$ & chi \pronounced{KI (as in hi)} \\
		$\lambda$, $\Lambda$ & lambda \pronounced{LAM-duh} & $\psi$, $\Psi$ & psi \pronounced{SIGH, or PSIGH} \\
		$\mu$ & mu \pronounced{MEW} & $\omega$, $\Omega$ & omega \pronounced{oh-MAY-guh} \\
		\bottomrule
	\end{tabular} \\[1.5ex]
	Capitals shown are the ones that differ from Roman capitals.
\end{center}

%----------------------------------------------------------------------------------------
%	GLOSSARY
%----------------------------------------------------------------------------------------

% The glossary needs to be compiled on the command line with 'makeglossaries main' from the template directory

\newglossaryentry{computer}{
	name=computer,
	description={is a programmable machine that receives input, stores and manipulates data, and provides output in a useful format}
}

% Glossary entries (used in text with e.g. \acrfull{fpsLabel} or \acrshort{fpsLabel})
\newacronym[longplural={Frames per Second}]{fpsLabel}{FPS}{Frame per Second}
\newacronym[longplural={Tables of Contents}]{tocLabel}{TOC}{Table of Contents}

\setglossarystyle{listgroup} % Set the style of the glossary (see https://en.wikibooks.org/wiki/LaTeX/Glossary for a reference)
\printglossary[title=Special Terms, toctitle=List of Terms] % Output the glossary, 'title' is the chapter heading for the glossary, toctitle is the table of contents heading

%----------------------------------------------------------------------------------------
%	INDEX
%----------------------------------------------------------------------------------------

% The index needs to be compiled on the command line with 'makeindex main' from the template directory

\printindex % Output the index

%----------------------------------------------------------------------------------------
%	BACK COVER
%----------------------------------------------------------------------------------------

% If you have a PDF/image file that you want to use as a back cover, uncomment the following lines

%\clearpage
%\thispagestyle{empty}
%\null%
%\clearpage
%\includepdf{cover-back.pdf}

%----------------------------------------------------------------------------------------

\end{document}
