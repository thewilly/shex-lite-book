\chapter{Project Communications}

\section{Open Source Community}
Regarding the open source community, from the beginning the work was considered as a
collaborative project where the community could debate, validate and contribute new
ideas to the project.

The \url{http://github.com/weso/shex-lite} repository has primarily been used as the central source of code. But
there is a parallel repository \url{http://github.com/weso/shex-lite-evolution} where there are records of some proposals that affect
the design of the system.

GitHubFlow\footnote{\url{https://guides.github.com/introduction/flow/}}, a variant of
GitFlow\footnote{\url{https://www.atlassian.com/es/git/tutorials/comparing-workflows/gitflow-workflow}} oriented to unlock the advance that occurs on many
occasions, has been used as a methodology to work. In this way a user can send an issue,
make the appropriate code modification and create a pull request that once accepted becomes
directly part of the most recent version of the system. An example of this is the
pull request SLP-0143 \url{https://github.com/weso/shex-lite/pull/143} where a community user implemented python code generation on
their own.

\section{Scientific Disclosure}
The research work of this dissertation has lead to papers that has been sent to different conferences.
The following papers are somehow derived from this dissertation:

\begin{enumerate}
    \item ShEx-Lite: Automatic generation of domainobject models from Shape Expressions.
    Guillermo Facundo Colunga, Alejandro González Hevia, Jose Emilio Labra Gayo, and Emilio Rubiera Azcona.
    \textit{19th International Semantiuc Web Conference. Posters and Demos Track.}
\end{enumerate}

\section{Community Meetings}
Also framed in the project, various meetings have been held with entities such as
the Dublin Core Metadata Initiative (\url{https://www.dublincore.org/}), Eric Prud'hommeaux (father of ShEx) or the management office of
the European Hercules ASIO project. During the Erick meeting the concept of the ShEx micro compact syntax and its Antlr transformation
where disscussed. During the DCMI meetings we discussed the aims of the project and they validate them. And with the ASIO management
office we disccussed how they will adopt out proposed solution as a production system for generating plain objects from their
schemas.
