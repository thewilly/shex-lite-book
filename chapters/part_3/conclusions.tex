\chapter{Conclusions}
\label{ch:conclusions}
The proposed analysis methods based on static syntactic and semantic analysis
techniques can be used to build different tools for software development, such
as development environments. Furthermore, after evaluating the results, we observe
that up to 90\% of ShEx users can benefit from the techniques described. And that
tools like RDFShape, YASHE or ShEx.js can adopt the techniques described here to
improve their static code analysis capabilities.

The analysis and proposed methods for transforming
schemas into models based on object-oriented technologies can be used to link
schemas made through Shape Expressions with business applications. An example
of this is the European project ASIO Hercules. Where the system proposed in
this document is used to automatically join an ontology with Java-based applications.
The results obtained in the ASIO Hercules European project have resulted in the paper \textit{"ASIO: a Research
Management System based on Semantic technologies"} sent to the CIKM\footnote{\url{https://www.cikm2020.org/}} 2020 conference.
In it, among other things, it explains the key role of the system proposed in this
dissertation to unite an Ontological Infrastructure with a Semantic Architecture.

The two solutions, the analyser and the translator, are embodied in a single system
called ShEx-Lite that includes the implementation of lexical, syntactic and semantic
analysers. And as code generation, translation into object-oriented languages. This
system has given as one of its results the article "ShEx-Lite: Automatic generation
of domain object models from Shape Expressions" that has been sent to the ISWC20\footnote{\url{https://iswc2020.semanticweb.org/}}
conference.

\section{Future Work}
This work opens new future lines of research that we plan to work on.
What follows is a brief description of such works.

\subsection{Implementation of New Analyses}
Currently we do not have an existing documentation on the most common errors
in ShEx and therefore it is difficult to generate new analyses, it is not known
what to look for. However, with the methods exploited in this project, you can
start, for example, to explore all open source schemes. From them obtain data
that serves to develop documentation on the most common errors or new analyses.

\subsection{Automatic Error Fixing}
The described system detects errors and locates them at a specific point in the
representation, through pattern recognition. The next step, once we have automatically
detected the errors, would be to develop a system that automatically corrects these
errors by means of transformations on the representations.

This opens the door to new lines of research, since the representation of the
schemas must be modified without affecting its schemas semantics.

\subsection{Machine Learning to Recognize Patters}
As explained in \cite{bigcode}, systems based on machine learning can be trained with
our intermediate schema representations, and then automatically classify the schemas
according to the training criteria.

\subsection{New Input Syntaxes for Translation to Aspect Oriented Languages}
As a line of future research but it is already underway. The Dublin Core research group
is working on defining a ShEx Micro based syntax with tabular representation, mainly
spreadsheets. This would allow to attract a much larger user base. As a proposal of
the group itself is to include this syntax as input to our ShEx-Lite system in order
to enhance all analytics and generation. For this we hold telematic meetings every
other Wednesday.

\subsection{New Target Languages}
ShEx-Lite only generates Java and Python code. However, adding the ability to generate
in new programming languages would increase the ability to integrate with more systems.