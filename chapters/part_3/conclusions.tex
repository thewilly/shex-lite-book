\chapter{Conclusions}
\label{ch:conclusions}
The analysis methods proposed for shape expressions can be used to perform lexical,
syntactic and grammar analyzes that can be used to build tools such as development
environments or compilation servers for shape expressions. In addition, it has been
shown that shape expressions can be integrated with object-oriented programming
languages. This proposal is materialized in ShEx-Lite, an infrastructure proposed
as a tradoctor where the analysis phase includes all the proposed methods and where
the synthesis phase applies the proposed transformation methods.

After evaluating the ShEx-Lite system with respect to the rest of the identified systems,
it can be seen that the content of the representations that other systems use to carry
out their analyzes can be enriched, so that with a better representation, better
validation can be performed.

Another important aspect to emphasize is that after evaluation we realize that the
communication system for syntactic and semantic errors of other systems may benefit
from this work.


\section{Future Work}
Currently both proposed solutions are based on the reduced ShEx grammar,
therefore the first future work we identify is to be able to bring the philosophies
described in this work to the full ShEx grammar, so that the improvements described
can benefit all users of the language.

The next step would be to expand the range of the static analysis of shape expressions
so that it supports more elements of the grammar so that all the elements that make
up a shape, their dependencies and relationships can be analyzed in much more detail.

\begin{itemize}
    \item One of the next steps is to adapt the proposed solution to generate code so
    that it reads a new form expressions syntax oriented to tabular formats. For this
    proposal, regular meetings are being held with the Dublin Core Metadata Initiative
    team, which is the driving force behind this new syntax.

\end{itemize}