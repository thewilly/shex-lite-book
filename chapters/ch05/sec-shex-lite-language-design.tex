\section{ShEx-Lite Language Design}
\labsec{ch05-language-design}

In the previous section we just seen the analysis of the requirements for the
syntax that we will develop in next chapter, here, in this section, we will
design the syntax from an abstract point of view.

The requirements alredy left preaty clear the functionalities of the language,
if must allow to \textbf{define prefixes}, \textbf{define a base}, \textbf{define the start shape expression}
and to \textbf{define shape expressions}. Therefore our languge will be formed by a set of definitions.

It is very important aso, in order to meet the quality attributes that our syntax is based on Shape Expressions
Compact Syntax. That means if our syntax is an extrictly subset of \texttt{ShExC} the systems that already
work with ShEx will work with ShEx-Lite. This has been disccused in \sidenote{\url{https://github.com/weso/shex-lite-evolution/pull/1}}.

\subsection{Prefix Definition}
The prefix definition will follow the same specification as in ShEx, that means that will assign an IRI to an String Identifier.
In ShEx it is allowed to re-assign a new IRI value to an existing identifier prefix \textit{(also called prefix overrdide)},
but in ShEx-Lite, as it is horiented to users with lower technical skills, this is not allowed.
This way we improve the safety of the language and its discussion can be found in \sidenote{\url{https://github.com/weso/shex-lite-evolution/pull/16}}.

In \texttt{ShExC} the prefix definition is representented by the keyword \texttt{prefix} followed by the string identifier, which can
be empty, then the colon symbol and finally the IRI. Therefore the production would be:

\begin{center}
    \begin{verbatim}
PrefixDefinition :: 'PREFIX'  ID ':' IRI
    \end{verbatim}
\end{center}

\subsection{Base Definition}
The base definition defines the IRI that will be use as base for other relative IRIs that could appear in the schema. By default the base
already contains an IRI that points to a default location and it is allowed to declare only one base definition in the schema. This is also done
by safety and its discussion can be found in \sidenote{\url{https://github.com/weso/shex-lite-evolution/pull/16}}.

Also we will inherit the \texttt{ShExC} syntax and the base definition will have the following production:

\begin{center}
    \begin{verbatim}
BaseDefinition :: 'BASE' IRI
    \end{verbatim}
\end{center}

\subsection{Start Definition}
The start definition does not have any implication within an scheme, its functionality is to indicate at validation time the default shape expression to use
if no other shapoe is indicated. As a difference with ShEx and in order to increase language safety we do not allow the start declaration to appear more than once.
This is an open discussion in ShEx community, the start shape expression that it is used is only the last one defined and the definition has no functionality during
compilation therefore if two start definitions appear the first one is like it never existed.

We will also folloe the \texttt{ShExC} syntax for this element and its production will be:

\begin{center}
    \begin{verbatim}
StartDefinition :: 'START' '=' ShapeReference
    \end{verbatim}
\end{center}

The \textit{ShapeDefinition} production also includes another rule that is the \textit{ShapeReference}. This rule production is not important at thi point,
but the important part is that it represents a reference to a shape expression definition that must be defined within the schema, at any possition.

\subsection{Shape Expression Definition}
The final element of the definitions is the Shape Definition, a shape definition is defined as a set of triple expressions, where each triple
expression is formed by the \textit{property}, the \textit{constraint} and the \textit{cardinality}. In ShEx this element definition is much more
complex, allowing logical operation between triple expressions and even assignments. In our case we simplified this element to cover the needs described
during the analysis. Our production for Shape Definition would be:

\begin{center}
    \begin{verbatim}
ShapeDefinition :: ShapeIdentifier '{' TripleExpression+ '}'
    \end{verbatim}
\end{center}

Notice the \textit{ShapeDefinition} contains a \textit{ShapeIdentifier}, this identifier is another prodsuction and at the end it is an IRI, it must
be unique in the schema and it stores the non-empty set \textit{(represented by the + symbol)} of triple expressions.