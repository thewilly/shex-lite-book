\section{Simplifications of ShEx}
\labsec{ch03-shex-simplifications}

\subsection{The \textbf{S} language}

In 2019 at \sidecite{rdf-challenges} was defined a language called \textbf{S} as a simple abstract
language that captures both the essence of ShEx and SHACL. This is very relevant as this language
is intended to be the input of a theoretical abstract machine that will be used for graph validation
for both ShEx and SHACL. Also in the same paper the authors carefully describe the algorithm for the
translation from ShEx to S and from SHACL to S.

Although the theoretical abstract machine has not been implemented yet the intention of the WESO
Research Group, where this S language was defined, is to devote more efforts in to this project
during the 2021.

Other definition of an abstract language based on uniform schemas can be found at \sidecite{iovka-auto-shex-shacl}.
This language is focused on schemas inference rather on validation, but needs to be taken
into account as they also perform an abstraction of both ShEx and SHACL.

\subsection{ShExJ Micro Spec}
Recently the Doublin Core Team\sidenote{\url{https://dublincore.org/}} is working into an
specification that allows to define Shape Expressions in tabular formats. For this specification
they propose a simplification of the Shape Expressions JSON syntax that allows to define an
schema as a set of simple triple contratints. This specification is not official and has
not been validated yet but it is very importat for our work as we will also work in a
simplification of a syntax of ShEx.

\bigskip

And to the best of our knowledge and after the research process carried out for this
project no other language based on a subset of Shape Expressions has been designed nor implemented yet.