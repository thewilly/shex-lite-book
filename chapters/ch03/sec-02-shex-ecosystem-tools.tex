\section{ShEx Ecosystem Tools}
\labsec{ch03-shex-ecosystem}

We already know that ShEx and SHACL have been the two main technologies for RDF validation and some tools emerged around them, we thinks that some of them might benefit from ShEx-Lite. Here we introduce briefly those that had the biggest impact in the community.

\subsection{Validators}
Since the beginning of ShEx and SHACL as languages the RDF community started to build tools that take as input the schemas defined and validate graphs.

This kind of tools can benefit from SHEx-Lite from the point of view that new functionalities can be easily implemented and tested in the lite version of the language before even touching the stable releases of both tools. In the case of ShEx this is more obvious as ShEx-Lite and ShEx are both implemented in Scala and if good design principles are used functionalities can be just migrated and expanded for the rest of the language.

The most important validators are:

\subsubsection{Shaclex}
According to the Shaclex\sidenote{\url{https://github.com/weso/shaclex}} official website it is an Open Source Scala pure functional implementation of an RDF Validator that supports both Shape Expressions and SHACL. It was initially developed by Dr. Jose Emilio Labra Gayo and is being maintained by an active community on GitHub. It is used by different projects around the globe and its goal is to validate RDF graphs against schemas defined in Shape Expression or in SHACL.
This implementation of a ShEx validator is very important for us as ShEx-Lite is completly inspired by it and aims to transfer the sintactic an semantic validation enhancements to it.

\subsubsection{ShEx.js}
Another example of os a ShEx validator implementation is \texttt{ShEx.js} which is JavaScript based and also open source on GitHub. This implementation is very important for the ShEx community as they defined the serialization of the AST in this implementation as the abstract syntax of ShEx.


\subsection{IDEs}
In order to facilitate the task of writing schemas some engineers decide to implement specific IDEs for the Shape Expressions Language.

This tools will completely benefit from ShEx-Lite and there are currently collaborations in process. At the time they work with Shaclex, which is structured as a conventional compiler, but with the API architecture of ShEx-Lite IDEs can access directly to the syntactic and semantic modules so features like advances coloring syntax or incremental compilation are available.

\subsubsection{YASHE}
YASHE\sidenote{\url{https://github.com/weso/YASHE}} (Yet Another ShEx Editor), is a Shape Expressions IDE which started as a fork ofYASQE(which is based on SPARQL). This tool performs lexical and syntactic analysis of the content of the editor, thus offering the user a realtime syntactic error detector. It has features like: syntax highlighting, visual aid elements (tooltips) and autocomplete mechanisms. In addition, it offers a simple way of integrating into other projects.

\subsubsection{Protégé}
Protégé is a piece of software developed by the University of Stanford focused on ontology edition. During the last year they added support for Shape Expressions dition on their own software so they became another ShEx IDE.

\subsubsection{VSCode}
VSCode is a source code light-weight editor developed by Micorsoft and supported by Linux, macOS and Windows. By default this editor does not support any programming language, the way it works is with packages that the community develops and extends the functionality. One of those packages adds support for Shape Expressions Compact syntax and transforms VSCode into a ShEx IDE.
This plugin does not add semantic validation and it is a clear target to benefit from ShEx-Lite features.

\subsection{Others}
Other researches focused their efforts in to inferring schemas to existing data sets and creating tools to that evolved from ShEx in order to transform existing data.

\subsubsection{Shexer}
Shexer\sidenote{\url{https://github.com/DaniFdezAlvarez/shexer}} is a python library aimed to perform automatic automatic extraction of schemas in both ShEx and SHACL from an RDF input graph. That is if all the other tools take the schemas as the input and validate a graph with it, this tool takes a graph and from it it infers the schemas that it might contain. Its work is fully described in \sidecite{iovka-auto-shex-shacl, fernandez2016inference}.

\subsubsection{ShExML}
ShExML\sidenote{\url{https://github.com/herminiogg/ShExML}} is a language based on ShEx (not a simplification nor an abstraction of ShEx) that can map and merge heterogeneous data formats into a single RDF representation. The main idea behind this tool is written at \sidecite{shexml}.

\bigskip

An example of how this different tools can work together thanks to ShEx-Lite would be the following, illustrated at \reffig{shex-lite-shexer-integration}.
Wikidata currently holds millions of registers that do not have any schema that validates them. And they need to make consumer that represents the data in to an object domain model. Without any tool this is just almost impossible, but this shexer you can infer the schemas to ShEx-Lite syntax and with the ShEx-Lite compiler you can automatically create the object domain model in your favorite OOL.

\begin{figure*}[h!]
	\includegraphics{shex-lite-shexer-integration}
	\caption[ShEx-Lite integration with Shexer]{ShEx-Lite integration with Shexer for automatically generating java domain object models for the Wikidata schemaless existing data. This shoes the schemaless data from wikidata from which shape expressions are infeered by shexer and later transformed to java plain objects by means of ShEx-Lite so third party apllications can implement the domain model.}
	\labfig{shex-lite-shexer-integration}
\end{figure*}