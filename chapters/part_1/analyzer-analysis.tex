\chapter{Proposed Sintactic and Semantic Analyzer}
\label{ch:proposed-sin-sema-anal}
After analyzing the existing tools, we can see that different
aspects of existing technology can be improved, such as those
discussed in \cref{sec:anal-enhacements}. In this chapter we
model a proposal by means of software engineering techniques.

Within these techniques, the process we are going to follow
to model the proposal is first obtain the use cases through
the possible improvements detected in the previous chapter.
From these use cases we will extract the requirements. Once
the requirements have been extracted, we will proceed to
design the solution using diagrams.

We know that the system will be composed of at least a lexical
analyzer, a syntactic analyzer, a semantic analyzer and some
type of message manager to handle errors and warnings.

\section{Error Handler}
Of the improvements that we observe in \cref{sec:anal-enhacements}, those
that have to do with the error / warning management
system are 1, 2 and 6. The following diagram considers
these use cases in the error / warning management system.

\begin{figure}[h!]
    \includegraphics[scale=0.6]{images/err-hand-use-case.pdf}
    \centering
    \caption[Error hanlder use cases]{Error hanlder use cases.}
    \label{fig:err-hand-use-case}
\end{figure}

Thus, from the use cases mentioned in the previous section we
extract the following functional requirements.

\begin{figure}[h!]
    \includegraphics[width=\textwidth]{images/err-hand-reqf.pdf}
    \centering
    \caption[Error hanlder functional requirements]{Error hanlder functional requirements.}
    \label{fig:err-hand-reqf}
\end{figure}

From the use cases we can also extract the following interface requirements.

\begin{figure}[h!]
    \includegraphics[width=\textwidth]{images/err-hand-reqnf.pdf}
    \centering
    \caption[Error hanlder non functional requirements]{Error hanlder non functional requirements.}
    \label{fig:err-hand-reqnf}
\end{figure}

For the previous requirements we propose the following modelation for the error management system (\cref{fig:err-hand-diag}).

\begin{figure}[h!]
    \includegraphics[width=\textwidth]{images/err-hand-diagram.pdf}
    \centering
    \caption[Error hanlder component and class diagrams]{Error hanlder component and class diagrams.}
    \label{fig:err-hand-diag}
\end{figure}


\section{Lexical Analyzer}
The lexical analyzer is necessary to subsequently carry out syntactic and semantic analysis.
And therefore, although it is not contemplated as an improvement in
the previous chapter, we do have to include it in our proposed system. The following diagram
shows the expected use cases of a lexical analyzer in our context.

\begin{figure}[h!]
    \includegraphics[scale=0.6]{images/lex-use-case.pdf}
    \centering
    \caption[Lexical analyzer use cases]{Lexical analyzer use cases.}
    \label{fig:lex-use-case}
\end{figure}

Thus, from the use cases mentioned in the previous section we
extract the following functional requirements.

\begin{figure}[h!]
    \includegraphics[width=\textwidth]{images/lex-reqf.pdf}
    \centering
    \caption[Lexical analyzer functional requirements]{Lexical analyzer functional requirements.}
    \label{fig:lex-reqf}
\end{figure}

From the use cases we can also extract the following interface requirements.

\begin{figure}[h!]
    \includegraphics[width=\textwidth]{images/lex-reqnf.pdf}
    \centering
    \caption[Lexical analyzer non functional requirements]{Lexical analyzer non functional requirements.}
    \label{fig:lex-reqnf}
\end{figure}

For the previous requirements we propose the following modelation for the lexical analyzer.

\begin{figure}[h!]
    \includegraphics[width=\textwidth]{images/lex-diagram.pdf}
    \centering
    \caption[Lexical analyzer component and class diagrams]{Lexical analyzer component and class diagrams.}
    \label{fig:lex-diag}
\end{figure}

\section{Sintactic Analyzer}
The parser may not be so necessary if we are looking to improve existing systems, but it is
necessary to carry out the next step, semantic analysis. To others
in this step you can also propose some improvement, although less. The following diagram
shows the expected use cases for a system that wants to implement a sintactic validator.

\begin{figure}[h!]
    \includegraphics[scale=0.6]{images/sin-use-case.pdf}
    \centering
    \caption[Sintactic analyzer use cases]{Sintactic analyzer use cases.}
    \label{fig:sin-use-case}
\end{figure}

Thus, from the use cases mentioned in the previous section we
extract the following functional requirements.

\begin{figure}[h!]
    \includegraphics[width=\textwidth]{images/sin-reqf.pdf}
    \centering
    \caption[Sintactic analyzer functional requirements]{Sintactic analyzer functional requirements.}
    \label{fig:sin-reqf}
\end{figure}

From the use cases we can also extract the following interface requirements.

\begin{figure}[h!]
    \includegraphics[width=\textwidth]{images/sin-reqnf.pdf}
    \centering
    \caption[Sintactic analyzer non functional requirements]{Sintactic analyzer non functional requirements.}
    \label{fig:sin-reqnf}
\end{figure}

For the previous requirements we propose the following modelation for the sintactic analyzer.

\begin{figure}[h!]
    \includegraphics[width=\textwidth]{images/sin-diagram.pdf}
    \centering
    \caption[Sintactic analyzer component and class diagrams]{Sintactic analyzer component and class diagrams.}
    \label{fig:sin-diag}
\end{figure}

\section{Semantic Analyzer}
The semantic analyzer is key in our architecture since most of the improvements
that have to do with finding new types of errors can be identified through semantic
validations. The following diagram shows the expected use cases for a system that
wants to implement a semantic validator to solve the above problems.

\begin{figure}[h!]
    \includegraphics[scale=0.6]{images/sema-use-case.pdf}
    \centering
    \caption[Semantic analyzer use cases]{Semantic analyzer use cases.}
    \label{fig:sema-use-case}
\end{figure}

Thus, from the use cases mentioned in the previous section we
extract the following functional requirements.

\begin{figure}[h!]
    \includegraphics[width=\textwidth]{images/sema-reqf.pdf}
    \centering
    \caption[Semantic analyzer functional requirements]{Semantic analyzer functional requirements.}
    \label{fig:sema-reqf}
\end{figure}

From the use cases we can also extract the following interface requirements.

\begin{figure}[h!]
    \includegraphics[width=\textwidth]{images/sema-reqnf.pdf}
    \centering
    \caption[Semantic analyzer non functional requirements]{Semantic analyzer non functional requirements.}
    \label{fig:sema-reqnf}
\end{figure}

For the previous requirements we propose the following modelation for the semantic analyzer.

\begin{figure}[h!]
    \includegraphics[width=\textwidth]{images/sema-diagram.pdf}
    \centering
    \caption[Semantic analyzer component and class diagrams]{Semantic analyzer component and class diagrams.}
    \label{fig:sema-diag}
\end{figure}

\section{Full System Diagram}
After analyzing and designing each component ahopra we offer a
complete view of the entire integrated system. In addition
you can see that a new component appears, the symbol table.
This component can be any type of structure that fulfills the
expected basic functions of a symbol table.

\begin{figure}[h!]
    \includegraphics[scale=0.4]{images/full-diagram.pdf}
    \centering
    \caption[Complete system class diagram]{Complete system class diagram.}
    \label{fig:full-diag}
\end{figure}