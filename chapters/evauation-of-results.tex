\chapter{Evaluation of Results}
\label{ch:results-evaluation}

\section{Methodology}
In order to evaluate the proposed solutions to the two questions posed in \cref{ch:intro},
the following methodologies has been used.

\begin{enumerate}
    \item To evaluate how error detection has been improved, compare the number of actual errors found
    in a form by the tools detected and against the proposed solution.
    
    \item To evaluate how the error information system has been improved, the number of elements that
    make up the error messages of each existing tool and of our solution is compared. In addition,
    a survey is carried out on different users familiar with the existing tools.

    \item To evaluate to what extent we can translate shapes to domain object models we collect all the
    existing shapes in github, reduce the set to those that fit the micro compact syntax and try to
    generate objects for those that are syntactically and semantically valid. In this way we can
    approximate what percentage we can translate.
\end{enumerate}

\section{Dataset}
The previous methodologies will be used on a dataset of its own shape expressions. As it does not currently
exist, to the best of our knowledge, no dataset of shape expressions has been used as a sample GitHub.
On this platform we have collected all files with the \texttt{.shex} extension and licensed for use. In addition,
we have filtered and reduced to only those schemas that were expressed through ShEx's reduced grammar.

\section{Results}