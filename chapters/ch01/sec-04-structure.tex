\section{Structure}\labsec{ch01-structure}
The dissertation layout is as follows:\\

\begin{description}
	\item[Chapter 2] Indicates the state of the art of the existing RDF validation technologies, tools for processing Shape Expressions and other related projects.
	\item[Chapter 3] Describes the goals that the project aim to achieve after its execution and possible real-world applications.
	\item[Chapter 4] Contains a detailed initial planning and budget for the project, this is the designed planning followed during the execution of the project and the initial estimated budget.
	\item[Chapter 5] Gives a basic theoretical background that it is needed to fully understand the concepts explained in the following chapters.
	\item[Chapter 6] Provides a technical description of the design and implementation of the compiler itself. This includes, analysis, design, the technological stack choices, diagrams, implementation decisions and tests.
	\item[Chapter 7] Compares the initial planning developed in chapter 4 with the final one. This includes the genuine execution planning of the project and the reasons and events that modified the one from chapter 4.
	\item[Chapter 8] Summarizes the analysis and results given over the project, gives an outlook for future work continuing the development of the implemented solution. And includes the diffusion of results done during the project.
	\item[Chapter 9] Includes all the set of references used during this document. It is fully recommended to read them carefully and use them as source of truth for any doubt.
	\item[Chapter 10] Attaches every document related to the project and referenced from other chapters that has been developed during the project. Here we include detailed budget, system manuals, and other documents.
\end{description}
