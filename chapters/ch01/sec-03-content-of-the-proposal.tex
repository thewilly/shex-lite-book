\section{Content of the proposal}\labsec{ch01-content-of-the-proposal}

After \refsec{ch01-motivation} and \refsec{ch01-main-usage-scenarios} this section descibes the developed system to solve the deficiencies and different profile-users requests.

\begin{description}
  \item[First] A compiler for a language defined as a subset of the shape expressions language focused on helping the non-expert user on solving problems with their schemas.
  \item[Secondly] A functionality in this compiler, that allows to automatically create domain object models in object-oriented programming languages, from the defined schemas.
\end{description}

\subsection{ShEx-Lite Compiler}
This compiler works over a defined subset of the Shape Expressions Compact Syntax, defined at \sidecite{shexc} that allows expressing basic constraints. It is implemented with the paradigm "compiler as a library" and it is able to parse a schema, analyze it and generate the syntactic and semantic errors that the schema contains.

The ShEx-Lite Compiler is composed of the following components:

\subsubsection{Syntax analysis}
The syntax analysis phase covers the transformation of the input in to an Abstract Syntax Tree. That is, lex and parse the file, generate the parse tree, raise any errors or warnings and finally build the AST.

\subsubsection{Semantic analysis}
The semantic analysis covers the validation and transformation of the AST in to the SIL (ShEx-Lite Intermediate Language). Is during this stage where the AST gets validated, type-checked and transformed from a tree to a graph, is this graph the one that gets the name of SIL.




\subsection{Automatic generation of domain object models}
But by far, the biggest difference with existing tools, is the automatic generation of domain object models from the schemas defined.

The idea behind this is to enhance interoperability between object oriented languages \sidecite[-40pt]{oopl} and RDF systems. An example of this is the European Project ASIO Hércules \sidecite[-20pt]{hercules-um}, where the automatic transformation of schemas in to POJOs \sidecite{pojo} is the tool that joins the Semantic Architecture and the Ontology Infrastructure.

Also it is important to remark here that we are perfectly conscious about the fact that not every object oriented language allows to model exactly the same restrictions as types differ, therefore each OOL needs to validate or map the schema to a representation on the language whose meaning is the same, that is create the image of the schema in the corresponding language.
