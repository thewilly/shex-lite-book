\section{Main usage scenarios}\labsec{ch01-main-usage-scenarios}

\refsec{ch01-motivation} introduced some profiles that might benefit from using ShEx-Lite. We can find an example of this profiles in the Wikidata Community. Wikidata is formed by a multidisciplinar community whose aim is to introduce RDF data in to an open knowledge base used by other companies like Google Search. The only problem is that the introduced RDF data needs validation to ensure a minimum data quality, but the profiles that introduce the data, usually, are domain experts whose knowledge about computer science is limited. \todo{Extender ejemplo wikidata.}

Besides to this, a common problem is that some companies like Wikidata or even Universities use ShEx to define the constraints of the RDF data that they own. But then, when developing applications with object oriented languages they need to translate those schemas in to a domain model to support their data. Furthermore if the Shape Expressions used to validate their data changes for some reason they need to rewrite that domain model in the OOL again. \todo{Adornar un poco.}

Finally, from a ShEx developer point of view sometimes appears the need to try new features in a small playground that allow easy an fast testing, for example a feature that appeared after this project was implemented is to automatically generate documentation webpages for the schemas defined in ShEx, but the first target of this feature won’t be ShEx, will be ShEx-Lite as it is perfect for he proof of concept.

\todo{Enumerar tipos de erramientas que se beneficiarían.}
