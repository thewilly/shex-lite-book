\chapter{Proposed Translator}
\label{ch:proposed-translator}

This chapter focuses on proposing a solution to ODMTP.
First formalizing the solution. And then modelling it
with elements of software engineering such as use cases
and requirements.

\section{Translator Formalization}
As a solution to the previous chapter, this section focuses on
proposing a application $f : S' \rightarrow M$ such that
applied on a schema, results in a domain model based on
plain objects.

Lets define $f(S') = \begin{bmatrix}f'(s_1')\\ f'(s_2')\\ \vdots\\ f'(s_n')\end{bmatrix}$ and
$f'(s_i') = \begin{bmatrix}f''(e_1')\\ f''(e_2')\\ \vdots\\ f''(e_n')\end{bmatrix}$. Then $f''(e_i')$
is the application that maps a triple expression $e'$ from $N \times T_{g} \times \{(1,1),(0,\infty)\}$
to $N \times T_g$. To find such a function we will use the knowledge that we already have.
We know that $p$ has a direct mapping as it belongs to $N$, $T_g$ maps to $T_g$ if
the cardinality value is $(1,1)$ or $(1, \infty)$. And the cardinality is aggregated to the type
so its not needed to map it. Then we define the application $f''(e')$ as $f:(p,t,c) \in N \times T_g \times \{(1,1), (0,\infty)\} \rightarrow (n,t)\in N \times T_g$
and therefore,

\begin{equation}\label{eq:transformation}
f''(e_i')
\begin{cases}
    (p,Proy_{t_g}lst) & if \; c=(1,1) \\
    (p,List[Proy_{t_g}lst]) & if \; c=(0,\infty)
\end{cases}.
\end{equation}

This application's function is to transform a triple expression
into an annotated type property. Where the $Proy_{tg}lst$ represents
the projection of the generic type from the abstraction of languages
of representation of plain objects on to the language specific type.
\cref{fig:lst-diagram} illustrates how the same input can lead to
multiple types due to the specific translators, that perform the 
$Proy_{tg}lst$ operation.

\begin{figure}
    \includegraphics[scale=0.8]{images/lsc-diagram.pdf}
    \centering
	\caption[Different target types generated by specific translators]{Different target types generated by specific translators.}
    \label{fig:lst-diagram}
\end{figure}

\section{Translator modelling}
At this point we already have an abstraction of our system ready.
We know that you must implement the transformation function previously explained.
Now we will lower our abstraction one level.
For this we will model our system by means of software engineering techniques
such as use cases, requirements or class diagrams. First, we will use the use case
method to find the necessary functionality of our system.

\begin{figure}[h!]
    \includegraphics[scale=0.8]{images/trans-use-case.pdf}
    \centering
    \caption[Translator use cases]{Translator use cases.}
    \label{fig:trans-use-case}
\end{figure}

\cref{fig:trans-use-case} shows us that in our system we will, of course, have the functionality to
translate ShEx schemas to domain models based on plain objects.
But we can also see that the entry may be wrong.
This implies that there must be some kind of input validation.
In addition, an error management system is also necessary.
\cref{fig:trans-flow} illustrates a high level view of the flowchart that the translator follows.

\begin{figure}
    \includegraphics[width=\textwidth]{images/diagrama-flujo-traductor.pdf}
    \centering
    \caption[Translator high level flowchart]{Translator high level flowchart.}
    \label{fig:trans-flow}
\end{figure}

Also from the developed use cases we can extract the list of functional and
non-functional requirements (external interfaces) that our system must support.

\begin{figure}[h!]
    \includegraphics[width=\textwidth]{images/trans-reqf.pdf}
    \centering
    \caption[Translator functional requirements]{Translator functional requirements.}
    \label{fig:trans-reqf}
\end{figure}

\begin{figure}[h!]
    \includegraphics[width=\textwidth]{images/trans-reqnf.pdf}
    \centering
    \caption[Translator non functional requirements]{Translator non functional requirements.}
    \label{fig:trans-reqnf}
\end{figure}

Thus, for the previous use cases and requirements the implementation abstract diagram will be.

\begin{figure}[h!]
    \includegraphics[width=\textwidth]{images/trans-diagram.pdf}
    \centering
    \caption[Translator component and class diagrams]{Translator component and class diagrams.}
    \label{fig:trans-diag}
\end{figure}