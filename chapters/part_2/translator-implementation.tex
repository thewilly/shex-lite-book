\chapter{Proposed Implementation}
\label{ch:proposed-system}

Our solution is based on a code translator. In the end, a translator is still a type of compiler \cref{fig:compiler-stages},
where we have the analysis and synthesis phase. The analysis phase focuses on verifying that the input is correct and on
making the necessary transformations. While the synthesis starts from a high quality structure and performs the
appropriate transformations to reach the target representation. In the case of our solution we have a difference
and that is that we do not have a single target but multiple ones (\cref{fig:translator}).

In addition to this, in \cref{ch:proposed-implementation}, we have already developed a system capable of analysing,
validating and transforming source code so that we obtained an intermediate language made up of a
high-quality graph. Therefore in our translator we will reuse the lexical, syntactic and semantic
analyser from \cref{ch:proposed-implementation}, that makes the front end of our translator.
And then, what we will truly develop in this section, is the back-end of the translator.

\begin{figure}
    \includegraphics{images/translator.pdf}
    \centering
	\caption[Translator generic structure]{Translator generic structure.}
    \label{fig:translator}
\end{figure}

\section{Translator Back-end}
In our solution, the back-end of the translator, also called the synthesis phase,
fulfils two main functions. On one hand, it analyses the intermediate language
in search of some specific incompatibility with the target language. And on the
other hand, it generates the specific code for the target language.

To perform the analysis of the intermediate language and through the Visitor pattern,
a graph path is implemented that validates that no node or set of nodes
violate the restrictions obtained in \cref{eq:restriction}. The visitor that is implemented is
completely reused from the one developed for the Syntax or Semantic analysis in \cref{ch:proposed-implementation}.

For code generation, it is proposed to carry out an implementation of the Visitor pattern for each of the specific code generators,
each one of the specific implementations will perform the transformation function described in \cref{eq:transformation},
without prejudice to the fact that each specific code generator may have more implementations of the visitor associated.
\cref{fig:class-diagram-cg} illustrates this situation where the Java code generation visitor actually contains four more visitors
one for each language specific level of the plain objects.

\begin{figure}
    \includegraphics[width=\textwidth]{images/class-diagra-cg.pdf}
    \centering
	\caption[Class diagram of the code generation visitors]{Class diagram of the code generation visitors.}
    \label{fig:class-diagram-cg}
\end{figure}

And as proof of concept of the structure proposed in the previous section, we implemented a system that starts from the one
developed in \cref{sec:anal-implementation} and is capable of generating code from schemas, checking that they are valid.
This solution follows structructure developed in this system and more precisely \cref{fig:class-diagram-cg}. Moreover this
developed system offers a CLI tool (\cref{fig:menu-tool}). In this tool the users can define multiple options as \texttt{--java-pkg=STRING} which if
present will trigger the java code generation and will generate the target object in the specified package.

For example, for the input \texttt{java -jar shexlc.jar --java-pkg=demo person.shexl} where the \texttt{person.shexl}
file corresponds to the schema defined at \cref{fig:shex-translate-small} \texttt{ShEx-Lite} would generate a single java
class with the code that appears at the \texttt{Person.java} file, also in \cref{fig:shex-translate-small}.
This implemented system will be used for evaluating the proposed solution.

\begin{figure}
    \includegraphics[width=\textwidth]{images/shexlc-menu.PNG}
    \centering
	\caption[CLI menu of \texttt{ShEx-Lite} CLI tool]{CLI menu of \texttt{ShEx-Lite} CLI tool.}
    \label{fig:menu-tool}
\end{figure}

\section{Tests}
In order to validate the translator, it is no longer possible to rely on the tests previously
defined by the ShEx specification. Therefore, synthetic tests have been designed for this purpose.
These synthetic tests prove that each extreme case of the code generation system works. In addition,
it is also proved that for each specific language of the code generation the type projection is correct.
In addition, all these tests are run continuously on GitHub on Windows, Linux and Mac OS platforms.
See \cref{ch:github} to see the CI configuration.

\section{Generated Objects}
In this section we will give real examples of use cases in which the proposed system has been used to generate objects,
in addition we will study the objects in order to estimate their quality.

\subsection{Real (Hercules ASIO European Project)}
In the framework of the European project Hercules ASIO, financed with FIVE MILLION FOUR HUNDRED SIXTY-TWO THOUSAND SIX HUNDRED
euros, the system described is used to link two parts of the architecture, the ontological infrastructure and the semantic 
architecture. The Hercules project tries to find a solution based on linked data to manage the research framework
in Spanish universities. Some examples of the objects generated in this project are \cref{fig:example-2} and \cref{fig:example-3}.

\begin{center}
	\noindent\begin{minipage}[t]{.4\textwidth}
		\begin{lstlisting}[frame=topline,numbers=left,firstnumber=0,title=\scriptsize\texttt{University.shexl}, basicstyle=\ttfamily\scriptsize]{a}
# Prefixes...
asio:University {
  rdfs:name xsd:string ;
  rdfs:county   xsd:string ;
  rdfs:location xsd:String ;
  asio:hasRector
    @asio:UniversityStaff ;
	...
}
		\end{lstlisting}
	\end{minipage}\hfill
	\begin{minipage}[t]{.5\textwidth}
		\begin{lstlisting}[language=Java, frame=t,numbers=left,firstnumber=0,title=\scriptsize\texttt{University.java}, basicstyle=\ttfamily\scriptsize]{b}
// Imports...
public class University {
  private String name ;
  private String country ;
  private String location ;
  private asio.UniversityStaff hasRector ;
  ...
  // Constructor...
  // Getters and Setters...
}
		\end{lstlisting}
	\end{minipage}
	\captionof{figure}{Schema modelling a \texttt{University} in \texttt{shexl} syntax to the left. And the \texttt{ShEx-Lite} generated code in \texttt{Java} to the right.}
	\label{fig:example-2}
\end{center}

\begin{center}
	\noindent\begin{minipage}[t]{.4\textwidth}
		\begin{lstlisting}[frame=topline,numbers=left,firstnumber=0,title=\scriptsize\texttt{Researcher.shexl}, basicstyle=\ttfamily\scriptsize]{a}
# Prefixes...
asio:Researcher {
  rdfs:name xsd:string ;
  rdfs:surname   xsd:string ;
  rdfs:id xsd:integer ;
  rdfs:orcid xsd:string ;
  rdfs:publications
    @asio:AcademicPublication * ;
	...
}
		\end{lstlisting}
	\end{minipage}\hfill
	\begin{minipage}[t]{.5\textwidth}
		\begin{lstlisting}[language=Java, frame=t,firstnumber=0,numbers=left,title=\scriptsize\texttt{Researcher.java}, basicstyle=\ttfamily\scriptsize]{b}
// Imports...
public class University {
  private String name ;
  private String surname ;
  private int id ;
  private String orcid ;
  private List<asio.AcademicPublication>
    publications ;
  ...
  // Constructor...
  // Getters and Setters...
}
		\end{lstlisting}
	\end{minipage}
	\captionof{figure}{Schema modelling a \texttt{Researcher} in \texttt{shexl} syntax to the left. And the \texttt{ShEx-Lite} generated code in \texttt{Java} to the right.}
	\label{fig:example-3}
\end{center}

\subsection{Synthetic (Generated)}
In addition to the actual use case mentioned above, different generations of synthetically generated objects have been made to validate that the generation is correct.
\cref{fig:example-4} illustrates a generated schema that contains all the possible types that our solution can represent in any object oriented language.

\begin{center}
	\noindent\begin{minipage}[t]{.4\textwidth}
		\begin{lstlisting}[frame=topline,numbers=left,firstnumber=0,title=\scriptsize\texttt{Synthetic.shexl}, basicstyle=\ttfamily\scriptsize]{a}
# Prefixes...
a:b {
	:c xsd:string ;
	:d xsd:integer ;
	:e xsd:float ;
	:e xsd:boolean ;
	:f @a:b ;
}
		\end{lstlisting}
	\end{minipage}\hfill
	\begin{minipage}[t]{.5\textwidth}
		\begin{lstlisting}[language=Java, frame=t,numbers=left,firstnumber=0,title=\scriptsize\texttt{Synthetic.java}, basicstyle=\ttfamily\scriptsize]{b}
package a;
// Imports...
public class B {
	private String c;
	private int d;
	private int e;
	private a.B f;
	// Constructor...
	// Getters and Setters...
}
		\end{lstlisting}
	\end{minipage}
	\captionof{figure}{Synthetic schema in \texttt{shexl} syntax to the left. And the \texttt{ShEx-Lite} generated code in \texttt{Java} to the right.}
	\label{fig:example-4}
\end{center}